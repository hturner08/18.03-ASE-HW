\documentclass[11pt,a4paper,titlepage]{article}
\usepackage[a4paper]{geometry}
\usepackage[utf8]{inputenc}
\usepackage[english]{babel}
\usepackage{lipsum}

\usepackage{amsmath, amssymb, amsfonts, amsthm, fouriernc, mathtools,commath}
% mathtools for: Aboxed (put box on last equation in align envirenment)
\usepackage{microtype} %improves the spacing between words and letters

\usepackage{graphicx}
\graphicspath{ {./pics/} {./eps/}}
\usepackage{epsfig}
\usepackage{epstopdf}

%%%%%%%%%%%%%%%%%%%%%%%%%%%%%%%%%%%%%%%%%%%%%%%%%%
%% COLOR DEFINITIONS
%%%%%%%%%%%%%%%%%%%%%%%%%%%%%%%%%%%%%%%%%%%%%%%%%%
\usepackage[svgnames]{xcolor} % Enabling mixing colors and color's call by 'svgnames'
%%%%%%%%%%%%%%%%%%%%%%%%%%%%%%%%%%%%%%%%%%%%%%%%%%
\definecolor{MyColor1}{rgb}{0.2,0.4,0.6} %mix personal color
\newcommand{\textb}{\color{Black} \usefont{OT1}{lmss}{m}{n}}
\newcommand{\blue}{\color{MyColor1} \usefont{OT1}{lmss}{m}{n}}
\newcommand{\blueb}{\color{MyColor1} \usefont{OT1}{lmss}{b}{n}}
\newcommand{\red}{\color{LightCoral} \usefont{OT1}{lmss}{m}{n}}
\newcommand{\green}{\color{Turquoise} \usefont{OT1}{lmss}{m}{n}}
%%%%%%%%%%%%%%%%%%%%%%%%%%%%%%%%%%%%%%%%%%%%%%%%%%




%%%%%%%%%%%%%%%%%%%%%%%%%%%%%%%%%%%%%%%%%%%%%%%%%%
%% FONTS AND COLORS
%%%%%%%%%%%%%%%%%%%%%%%%%%%%%%%%%%%%%%%%%%%%%%%%%%
%    SECTIONS
%%%%%%%%%%%%%%%%%%%%%%%%%%%%%%%%%%%%%%%%%%%%%%%%%%
\usepackage{titlesec}
\usepackage{sectsty}
%%%%%%%%%%%%%%%%%%%%%%%%
%set section/subsections HEADINGS font and color
\sectionfont{\color{MyColor1}}  % sets colour of sections
\subsectionfont{\color{MyColor1}}  % sets colour of sections

%set section enumerator to arabic number (see footnotes markings alternatives)
\renewcommand\thesection{\arabic{section}.} %define sections numbering
\renewcommand\thesubsection{\thesection\arabic{subsection}} %subsec.num.

%define new section style
\newcommand{\mysection}{
\titleformat{\section} [runin] {\usefont{OT1}{lmss}{b}{n}\color{MyColor1}} 
{\thesection} {3pt} {} } 

%%%%%%%%%%%%%%%%%%%%%%%%%%%%%%%%%%%%%%%%%%%%%%%%%%
%		CAPTIONS
%%%%%%%%%%%%%%%%%%%%%%%%%%%%%%%%%%%%%%%%%%%%%%%%%%
\usepackage{caption}
\usepackage{subcaption}
%%%%%%%%%%%%%%%%%%%%%%%%
\captionsetup[figure]{labelfont={color=Turquoise}}

%%%%%%%%%%%%%%%%%%%%%%%%%%%%%%%%%%%%%%%%%%%%%%%%%%
%		!!!EQUATION (ARRAY) --> USING ALIGN INSTEAD
%%%%%%%%%%%%%%%%%%%%%%%%%%%%%%%%%%%%%%%%%%%%%%%%%%
%using amsmath package to redefine eq. numeration (1.1, 1.2, ...) 
%%%%%%%%%%%%%%%%%%%%%%%%

%set box background to grey in align environment 
\usepackage{etoolbox}% http://ctan.org/pkg/etoolbox
\makeatletter
\patchcmd{\@Aboxed}{\boxed{#1#2}}{\colorbox{black!15}{$#1#2$}}{}{}%
\patchcmd{\@boxed}{\boxed{#1#2}}{\colorbox{black!15}{$#1#2$}}{}{}%
\makeatother
%%%%%%%%%%%%%%%%%%%%%%%%%%%%%%%%%%%%%%%%%%%%%%%%%%




%%%%%%%%%%%%%%%%%%%%%%%%%%%%%%%%%%%%%%%%%%%%%%%%%%
%% DESIGN CIRCUITS
%%%%%%%%%%%%%%%%%%%%%%%%%%%%%%%%%%%%%%%%%%%%%%%%%%
\usepackage[siunitx, american, smartlabels, cute inductors, europeanvoltages]{circuitikz}
%%%%%%%%%%%%%%%%%%%%%%%%%%%%%%%%%%%%%%%%%%%%%%%%%%



\makeatletter
\let\reftagform@=\tagform@
\def\tagform@#1{\maketag@@@{(\ignorespaces\textcolor{red}{#1}\unskip\@@italiccorr)}}
\renewcommand{\eqref}[1]{\textup{\reftagform@{\ref{#1}}}}
\makeatother
\usepackage{hyperref}
\hypersetup{colorlinks=true}

\DeclareMathOperator{\sininv}{sin^{-1}}
%%%%%%%%%%%%%%%%%%%%%%%%%%%%%%%%%%%%%%%%%%%%%%%%%%
%% PREPARE TITLE
%%%%%%%%%%%%%%%%%%%%%%%%%%%%%%%%%%%%%%%%%%%%%%%%%%
\title{\blue Differential Equations \\
\blueb ASE Assignment Solutions}
\author{Herbert Turner}
\date{\today}
%%%%%%%%%%%%%%%%%%%%%%%%%%%%%%%%%%%%%%%%%%%%%%%%%%



\begin{document}
\maketitle

\section{Part $1$}
\subsection{HW #1}
{Problem 1A-2
\begin{itemize}
\item[b.)] 1
\item[d.)] Although there are 4 constants in the initial expression, it can be rewritten as $ln(acx^2+(ad+bc)x + bd).$ This allows the number of necessary constants to be reduced to 3.
\end{itemize}
}
{Problem 1A-3b
\begin{align*}
\frac{dy}{dx} &= \frac{ye^x}{x} \\
\int_{1}^{x}\frac{1}{y}\,dy &= \int_{1}^{x}\frac{e^x}{x}\,dx \\
ln(y)\Big|_2^x &= \int_{1}^{x}\frac{e^x}{x}\,dx \\
ln(y(x)) \Big|_{1}^{y(x)}&= \int_{1}^{x}\frac{e^x}{x}\,dx \\
y(x) &= e^\int_{1}^{x}\frac{e^x}{x}\,dx
\end{align*}}
{Problem 1A-4a%
\begin{align*}
\frac{dy}{dx} &= \frac{xy+x}{y} \\
\int_{2}^{y(x)}\frac{y}{y+1}\,dy &= \int_{1}^{x}x,dx \\
\int_{2}^{y(x)}1-\frac{1}{y+1}\,dy &= \frac{x^2}{2}\Big|_2^x\\
y-ln(y+1)\Big|_2^x &= \frac{y^2}{2} - 2 \\
y(x) - ln(y(x)+1) &= \frac{x^2}{2} -2
\end{align*}
}
\label{sec:q1sec}
{Problem 1A-5b
\begin{align*}
\int_{}^{}\frac{1}{\sqrt{2}1-v^2}\,dv &= \int_{}^{}\frac{1}{x}\,dx \\
\sininv{v} &= ln x + C \\ 
v &= sin(ln x + c)
\end{align*}
}
\subsubsection{Problem 1D-1a}
For the point of a tangency to be the midpoint of the segment in the first quadrant, the slope of the line must be -y/x. We can now setup a differential equation and solve for y.
\begin{align*}
\frac{dy}{dx} &= \frac{-y}{x} \\
y &= \frac{c}{x}
\end{align*}

\subsubsection{Problem 1D-5}
\begin{align*}
\frac{dT}{dt} &= A(T-20)\\
\int_{}^{}\frac{1}{A(T-20)} &= \int_{}^{}\,dt \\
\frac{1}{A}ln(T-20) &= t + C \\
T-20 &= B*e^{At} \\
T &= B*e^{At} + 20 \\
\end{align*}
Using the initial conditions, we can solve for B at t = 0, and for A at t = 5
\begin{align*}
100 &= B + 20   &   B &= 80e^{5A} + 20\\
B &= 80         &   60 &= 80e^{5A}\\
&               &  A&=\frac{1}{5}ln(3/4)\\
\end{align*}
\subsubsection{Problem 1D-6b}
\begin{align*}
mg - kv^2 &= \frac{dv}{dt}m \\
\frac{-k}{m}(v^2-a^2) &= \frac{dv}{dt}\\
\int_{}^{}\frac{-k}{m}\,dt &= \int_{}^{}\frac{1}{v^2-a^2}\,dv\\
\frac{-k}{m}t + C &= \int_{}^{}\frac{1}{2a}(\frac{1}{v-a}-\frac{1}{v+a})\,dv\\
\frac{-2ak}{m}t + C &= ln|\frac{v-a}{v+a}|\\
\shortintertext{We can now solve for C since we know at t=0, v= 0}
0 + C &= ln 1\\
C  &= 0\\
\shortintertext{then}
\frac{a-v}{a+v} &= e^{\frac{-2akt}{m}}\\
v &= a\frac{1-e^{\frac{-2akt}{m}}}{1+e^{\frac{-2akt}{m}}}
\end{align*}
%%% END SECTION 1 %%%%%%%%%%%%%%%%%%%%%%%%%%%%%%%%%%%%%%%
\subsection{HW #2}
\subsubsection{1B-5a}
\begin{align*}
\frac{dy}{dx} + \frac{2}{x}y &= 1 \\
\rho(x) &= e^{2ln|x|}\\
e^{2ln|x|}\frac{dy}{dx} + \frac{2}{x}ye^{2ln|x|} &= e^{2ln|x}\\
e^{2ln|x|}y &= \int_{}^{}e^{2ln|x|}\,dx \\
yx^2 &= \int_{}^{}x^2\,dx \\
y &= \frac{x}{3} + \frac{C}{x^2}
\end{align*}
\subsubsection{1B-6}
\begin{align*}
\frac{dx}{dt} + ax &= r(t) \\
\rho(t) &= e^{at} \\
e^{at}x(t) + C &= int_{}^{}e^{at}r(t)\,dt \\
x(t) &= \frac{\int_{}^{}e^{at}r(t)\,dt - C}{e^{at}}\\
\shortintertext{Because at the given limit the denominator and numerator approach infinity, we can use l'hopital's rule and take the derivative of both sides:}
\lim{t\to\infty} x(t) &= \lim{t\to\infty} \frac{e^at*r(t)}{ae^{at}} \\
&= \lim{t\to\infty} \frac{r(t)}{a}\\
& = 0
\end{align*}
\subsubsection{1B-7}
\begin{align*}
\frac{dy}{dx} &= \frac{y}{y^3+x} \\
\frac{dx}{dy} &= y^2 + \frac{x}{y} \\
\frac{dx}{dy} - \frac{x}{y} &= y^2 \\
\rho(y) &= e^{-ln|y|} \\
&= \frac{1}{y}\\
\frac{1}{y} \frac{dx}{dy} - \frac{1}{y^2}x &= y \\
\frac{x}{y} &= \frac{y^2}{2} + C \\
x &= \frac{y^3}{2} + Cy
\end{align*}
\subsubsection{1D-4}
\begin{itemize}
    \item[a.)] 
    \begin{align*}
    \frac{dB}{dt}  &= \lambda _1 A_0e^{-\lambda _1 t} - \lambda _2B(t)\\
    \rho(t) &= e^{\lambda _2t}\\
    e^{\lambda _2t}B(t) &= \int_{}^{}\lambda_1A_0e^{(\lambda_2 - \lambda_1)t}\,dt + C\\
    B(t) &= \frac{\lambda_1A_0}{\lambda_2-\lambda_1}e^{-\lambda_1t}+\frac{C}{e^{\lambda_2t}}\\
        \shortintertext{Solving for C at the initial condition t = 0, we find}
    C &= B(0) - \frac{\lambda_1A_0}{\lambda_2-\lambda_1}
    \end{align*}
    \item[b.)]
    \begin{align*}
        \frac{dB}{dt} &= 0 \\
        e^{-t} &= \frac{A_0}{2(A_0+B(0)} &= e^{-t} \\
        t &= -ln(\frac{A_0}{2(A_0 + B_0)})\\
    \end{align*}
    If $ A <= 2B_0 $, there is no solution(max of t is at 0).
\end{itemize}
\subsubsection{1C-1}
%%%%%%%%%TODO%%%%%%%%%%%%%%%%%%%%%%%%%%%
\begin{itemize}
    \item[d.)]
    \item[e.)]
\end{itemize}
\subsection{HW#3}
\subsection{HW#4}
\sfrac{dy}{dx}ubsection{HW#5}
\subsection{HW#6}
\subsection {HW#7}
\subsection{HW#8}
\subsection{HW#9}
\subsection{HW#10}

\section{Question 2 Solution}{
\subsection{Section 1}{
Consider the circuit on Figure~\ref{fig:q2fig1b}. Now\footnote{Inductor is Short} applying current divider technique one could easily calculate the following currents:\par
\begin{align*}
i_1 = \dfrac{40}{500+2k||6k}\cdot\dfrac{2k}{500+2k||6k}\\ \\
i_2 = \dfrac{40}{500+2k||6k}\cdot\dfrac{6k}{500+2k||6k}\\ \\
\Aboxed{i_1(0^-)=20\,\si{mA} \qquad i_2(0^-)=60\,\si{mA}}
\end{align*}
%%%%%%%%%%%%%%%%%%%%%%%%%%%%%%%%%%%%%%%%%%%%%%%%%%
\ctikzset {bipoles/length=.8cm}
\begin{figure}[!htb]
\centering
\begin{subfigure}{.5\textwidth}
\begin{circuitikz}[scale =.6]\draw
(0,0) to [voltage source = $40\,\si{\volt}$] (0,3)
to [R = $500~\si{\ohm}$] (2,3)
to [opening switch, l_= {t>0}] (3,3) -- (4,3)
node[anchor=south]{$v_x$}
to [R = $6~\si{\kohm}$, i>_=$i_1$, *-] (7,3)
to [L, l_= $400~\si{m\henry}$] (7,0) -- (0,0)
(4,3) to [R, l_= $2~\si{\kohm}$, i>_=$i_2$] (4,0)
;\end{circuitikz}
\caption{\green \,Superposed Circuit}
\label{fig:q2fig1a}
\end{subfigure}
\begin{subfigure}{.5\textwidth}
\begin{circuitikz}[scale =.6]\draw
(0,0) to [voltage source = $40\,\si{\volt}$] (0,3)
to [R = $500~\si{\ohm}$] (2,3) -- (4,3)
to [R = $6~\si{\kohm}$, i>_=$i_1$] (6,3)
to [L, l= $400~\si{m\henry}$] (6,0) -- (0,0)
(3,3) node[anchor=south]{$v_x$}
to [R, l_= $2~\si{\kohm}$, i>_=$i_2$,*-] (3,0)
;\end{circuitikz}
\caption{\green \,Past Time Circuit}
\label{fig:q2fig1b}
\end{subfigure}
\begin{subfigure}{.4\textwidth}
\begin{circuitikz}[scale =.6]\draw
(0,0) to [R = $8\,\si{\kohm}$, i_<=$i_2$,-*] (0,3) -- (2,3) 
node[anchor=south]{$v_L$}
to [L = $400\,\si{m\henry}$, i>^=$i_1$,*-] (2,0) -- (0,0)
;\end{circuitikz}
\caption{\green \,Future Time Circuit}
\label{fig:q2fig1c}
\end{subfigure}
\caption{\green Question 2 Circuit}
\label{fig:q2fig1}
\end{figure}
}
%%% END SUBSECTION 1 %%%%%%%%%%%%%%%%%%%%%%%%%%%%%%%%%%%%%%
\subsection{Section 2}{
Assuming continuity in the inductor current $i_1(0^-)=i_1(0^+)=20\,\si{\mA}$ on Figure~\ref{fig:q2fig1c}, where $i_1=-i_2$, thus, setting $i_1(0^+)=-i_2(0^+)=20\,\si{\mA}$ we get the currents
\begin{align*}
\Aboxed{i_1(0^+)=20\,\si{mA} \qquad i_2(0^+)=-20\,\si{\mA}}
\end{align*}
}
%%% END SUBSECTION 2 %%%%%%%%%%%%%%%%%%%%%%%%%%%%%%%%%%%%%%
\subsection{Section 3}{It's obvious, that circuits \ref{fig:q2fig1b} and \ref{fig:q2fig1c} are equivalent to the one on Figure~\ref{fig:q1fig} (Page~\pageref{fig:q1fig}) with $\tau=50~\si{\ms}$ and $I_0=V_0/R_{eq}=20~\si{\mA}$:
\begin{align}
\Aboxed{\dot{i_L}(t)+\tfrac{R}{L}i_L(t)&=\tfrac{R}{L}I_0 \quad,\, t>0}
\end{align}
}
%%% END SUBSECTION 3 %%%%%%%%%%%%%%%%%%%%%%%%%%%%%%%%%%%%%%
\subsection{Section 4}{Adopting ZIR\footnote{Since, there is no source in the circuit} solution~\eqref{eq:7} 
\begin{align*}
\Aboxed{i_1(t)&=i_1(0^+)\exp\left({\dfrac{-t}{\tau}}\right) \quad, t>0} \\
\Aboxed{i_2(t)&=-i_1(t)}
\end{align*}
}
%%% END SUBSECTION 3 %%%%%%%%%%%%%%%%%%%%%%%%%%%%%%%%%%%%%%
}\label{sec:q2sec}
%%% END SECTION 2 %%%%%%%%%%%%%%%%%%%%%%%%%%%%%%%%%%%%%%%%



\section{Question 3 Solution}{
\subsection{Section 1}{
The voltage drop on the capacitor\footnote{Open-Circuited Capacitor } on Figure~\ref{fig:q3fig1b} as follows
\begin{align}
\Aboxed{v_C(0^-)&=15~\si{\mA} \cdot 2.4~\si{\kohm}=36~\si{\volt}  \label{eq:8}}
\end{align}
\ctikzset {bipoles/length=.8cm}
\begin{figure}[!htb]
\begin{subfigure}{\textwidth}
\begin{circuitikz}[scale =.8]\draw
(0,0) to [current source = $15\,\si{\mA}$] (0,3) -- (2,3)
to [R, l= $2.4\,\si{\kohm}$] (2,0) -- (4,0)
to [C,l_=$0.25\,\si{\micro\farad}$] (4,2.5)
to[short, -o](3.3,3.3)
(4,3)node[anchor=south]{t=0}
node[ocirc] (A) at (3,3) {}
node[ocirc] (B) at (5,3) {}
(B) to [open, v=${}$] (A)
(5,3)to[short, o-](6,3)--(6,5)
to [R=$25\,\si{\kohm}$] (8,5)--(8,3)
to [cI,l_=$\alpha \cdot v_{\phi}$] (6,3)
(2,3)to[short,-o](3,3)
(8,3)--(9,3)
to [R, l_= $15~\si{\kohm}$,v^=$v_{\phi}$] (9,0) -- (0,0)
;\end{circuitikz}
\caption{\green \,Superposed Circuit}
\label{fig:q3fig1a}
\end{subfigure}
\begin{subfigure}{.5\textwidth}
\begin{circuitikz}[scale =1]\draw
(0,0) to [current source = $15~\si{\mA}$] (0,2) -- (2,2)
to [R, l_= $2.4~\si{\kohm}$] (2,0)
(2,2) -- (4,2) 
node[ocirc] (A) at (4,0) {}
node[ocirc] (B) at (4,2) {}
(B) to [open, v^=${v_c}$] (A)
(0,0)--(4,0)
;\end{circuitikz}
\caption{\green \,Past Time Circuit}
\label{fig:q3fig1b}
\end{subfigure}
\begin{subfigure}{.4\textwidth}
\begin{circuitikz}[scale =.6]\draw
node[ocirc] (A) at (0,0) {}
node[ocirc] (B) at (0,3) {}
(B) to [open, v=${v_t}$] (A)
(0,3) to [short, i=${i_t}$](1,3)--(2.7,3)--(2.7,4.5) 
to [R, l_= $25~\si{\kohm}$] (6.5,4.5)--(6.5,2.2)
to [cI, l= $\alpha \cdot v_{\phi}$] (2.7,2.2)--(2.7,3)
(6.5,3)--(8,3) to [R, l_= $15~\si{\kohm}$,v^=$v_{\phi}$] (8,0)--(0,0)
;\end{circuitikz}
\caption{\green \,Future Time Circuit}
\label{fig:q3fig1c}
\end{subfigure}
\caption{\green Question 3 Circuit}
\label{fig:q3fig1}
\end{figure}
%%%%%%%%%%%%%%%%%%%%%%%%%%%%%%%%%%%%%%%%%%%%%%%%%%
}
%%% END SUBSECTION 1 %%%%%%%%%%%%%%%%%%%%%%%%%%%%%%%%%%%%%%
\subsection{Section 2}{
Replacing capacitor with a test voltage, as shown on Figure~\ref{fig:q3fig1c}, allows input resistance calculation $R_{in}=\dfrac{v_t}{i_t}$
\begin{align}
v_t-v_{\phi}&=25k\alpha \cdot v_{\phi} \nonumber \\
v_t=(1+25k\alpha) \cdot v_{\phi}&=15k(1+25k\alpha) \cdot i_t \nonumber \\
\Aboxed{R_{in}(C)&=15k(1+25k\alpha) \label{eq:10}}
\end{align}
}
%%% END SUBSECTION 2 %%%%%%%%%%%%%%%%%%%%%%%%%%%%%%%%%%%%%%
\subsection{Section 3}{Given circuit's time constant $\tau=R \cdot C=25~\si{ms}$ requires resistor $R=100~\si{\kohm}$. Setting into~\eqref{eq:10} leads to the value of $\alpha$-parameter
\begin{align*}
\Aboxed{\alpha \approx 2.26 \cdot 10^{-4}}
\end{align*}
}
%%% END SUBSECTION 3 %%%%%%%%%%%%%%%%%%%%%%%%%%%%%%%%%%%%%%
\subsection{Section 4}{As before\footnote{Due to circuits analogy} using ZIR solution~\eqref{eq:7} adapted for voltages, where $\tau=25~\si{ms}$ and initial condition of the capacitor\footnote{Continuity in voltage drop} $v_C(0^-)=v_C(0^+)=36~\si{\volt}$
\begin{align*}
\begin{cases}
\Aboxed{v_C(t)&=v_C(0^+)\exp\left({\dfrac{-t}{\tau}}\right) \quad, t\geqslant0} \\
\Aboxed{v_C(t)&=v_C(0^-) \hspace{5.1em}, t<0} \\
\end{cases}
\end{align*}
}
%%% END SUBSECTION 4 %%%%%%%%%%%%%%%%%%%%%%%%%%%%%%%%%%%%%%
}\label{sec:q3sec}
%%% END SECTION 3 %%%%%%%%%%%%%%%%%%%%%%%%%%%%%%%%%%%%%%%



\end{document}